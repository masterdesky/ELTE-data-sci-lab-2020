%----------------------------------------------------------------------------------------
%	SLIDE 1.
%----------------------------------------------------------------------------------------
\begin{frame}
\frametitle{Motivations and goals of topic}

\begin{itemize}
		\item<1-> Gábor's work was based on an already existing article on the problem\footnote<1->{Szüle, J., Kondor, D., Dobos, L., Csabai, I., \& Vattay, G. (2014). Lost in the City: Revisiting Milgram's Experiment in the Age of Social Networks. PloS one, 9(11), e111973.}
	\item<2-> Detect indications of the small-world phenomenon in US Twitter user data
	\begin{itemize}
		\item<2-> Mostly \textbf{short path lengths} in the graph/network of user nodes
		\item<2-> "Connections" are simply defined as users following other users
		\item<2-> Twitter has an extra information: \textit{Geographical data}
	\end{itemize}
	\item<3-> Explore the \textbf{navigability} of Twitter users' network
	\begin{itemize}
		\item<3-> A network is navigable, if some decentralized algorithm can find the path between two arbitrary nodes
		\item<3-> The decentralized algorithm of choice for this problem is the \textbf{Greedy algorithm}
	\end{itemize}	
\end{itemize}

\end{frame}